%! Author = thibaultchausson
%! Date = 23/07/2023

\subsection{Formalisation du message}\label{subsec:formalisation-du-message}

\textcolor{red}{A finir enfin à faire}

\subsubsection{L'AE en Réel}
\begin{itemize}
    \item \underline{Image 1 :} Jeux de société en cours dans le Foyer des Étudiant·e·s de Belfort, montrant une ambiance dynamique.
    \item \underline{Texte 1 :} \og Activité d'intégration: Jeux de société avec le club de l'AE le Troll Penché. \fg{}
    \item \underline{Image 2 :} Président de l'AE parlant des objectifs du semestre (Battle de soirées, clubs, investissement dans les lieux de vie).
    \item \underline{Image 2 :}
\end{itemize}

\subsubsection{Élection du PDF}
\begin{itemize}
    \item \underline{Image 1 :} Président de l'AE, Alexis BOULIGAND, introduisant le nouveau fonctionnement du PDF.
    \item \underline{Texte 1 :} \og Nouveau fonctionnement du PDF: Élections par listes et soirées. \fg{}
    \item \underline{Image 2 :} Informations sur le processus d'élection et les listes.
    \item \underline{Texte 2 :} \og Inscription jusqu'au 25 septembre, élections par les membres de l'AE. \fg{}
\end{itemize}

\subsubsection{Soirée Char d’Assos}
\begin{itemize}
    \item \underline{Image 1 :} Samuel, Benjamin et Elliot présentant le festival.
    \item \underline{Texte 1 :} \og Soirée Char d’Assos: Festival de musique à la Citadelle de Belfort, 28 septembre 2023. \fg{}
    \item \underline{Image 2 :} Problèmes techniques lors de la réalisation du Reel (micro, bruits).
    \item \underline{Texte 2 :} \og Défis techniques et succès: 66 likes sur 184 followers. \fg{}
\end{itemize}

\subsubsection{Présentation des lieux de vie de Montbéliard}
\begin{itemize}
    \item \underline{Image 1 :} Présentation générale des lieux par Énora.
    \item \underline{Texte 1 :} \og Découverte des avantages de l'AE à Montbéliard. \fg{}
    \item \underline{Image 2 :} Problèmes techniques avec le micro-cravate.
    \item \underline{Texte 2 :} \og Défis techniques: Retouches sonores avec Audacity. \fg{}
\end{itemize}

\subsubsection{Présentation des lieux de vie de Sevenans}
\begin{itemize}
    \item \underline{Image 1 :} Albane et Albin présentant la MDE.
    \item \underline{Texte 1 :} \og MDE de Sevenans: Un lieu de vie et d’activités dynamique. \fg{}
    \item \underline{Image 2 :} Difficultés liées au niveau sonore pendant les prises.
    \item \underline{Texte 2 :} \og Défis techniques: Gestion du bruit ambiant. \fg{}
\end{itemize}





\subsection{Les informations transmises}\label{subsec:les-informations-transmises}

\textcolor{red}{Justification des informations transmise, pourquoi le message, pourquoi ce choix technique}


\subsection{Choix de la direction artistique}\label{subsec:choix-de-la-direction-artistique}


\textcolor{red}{A FAIRE}




\subsection{Réalisation des vidéos}\label{subsec:realisation-des-videos}

\subsubsection{L'\gls{AE} en Réel}\label{subsubsec:ae-en-reel}

Ce Reel a été réalisé le dimanche 10 septembre 2023, pendant une activité de l'intégration : l'après-midi jeux de société co-organiser avec le club de l'\gls{AE} le Troll Penché.
Cette activité prenait place dans le Foyer des Étudiant·e·s de Belfort.
J'ai décidé de choisir ce moment pour tourner cette vidéo, car le Foyer était dynamique et vivant, ce qui n'est pas toujours le cas, dû à son emplacement un peu excentré des bâtiments d'enseignement de l'\gls{UTBM} Belfort.
Ce dynamisme, lui a permis de se montrer sous son meilleur jour auprès des étudiant·e·s de l'\gls{UTBM}, qui parfois peuvent en avoir une image négative.
Ainsi, cette ambiance a permis de découler sur l'image de l'\gls{AE} et donc montrer une association pleine de vie.

Outre le point précédent, l'objectif premier de ce Reel est de mettre une image sur la fonction de président de l'\gls{AE}.
Durant cette courte capsule, il a pu évoquer les principales ambitions de l'association pour ce semestre :
\begin{itemize}
    \item Battle de soirées, sans dire l'objectif premier de cette bataille, qui sera évoquée dans le Reel suivant (voir partie : \ref{subsubsec:election-pdf})
    \item Présentation et mise en avant des clubs
    \item Investissement sur les lieux de vie et les sites
\end{itemize}


La vidéo se termine en demandant une implication des membres pour qu'il·elle·s nous apportent des idées sur ceux qu'il·elle·s veulent voir de nouveaux.
À l'heure où je rédige ce document, deux idées ont été évoquées :
\begin{itemize}
    \item Achat d'une console nouvelle génération
    \item Achat d'un Google Chrome Cast
\end{itemize}

Le Reel peut se retrouver sur Instagram en \href{https://www.instagram.com/reel/CxGShAusxDq/?utm_source=ig_web_copy_link&igshid=MzRlODBiNWFlZA==}{cliquant ici}.


\subsubsection{Élection du \gls{PDF}}\label{subsubsec:election-pdf}


Ce Reel a été tourné le même jour que le Reel de présentation de l'AE (voir partie \ref{subsubsec:ae-en-reel}), ainsi, nous retrouvons les mêmes motivations qu'évoqué précédement.

Ce Reel a pour but de présenter le nouveau fonctionnement du \gls{PDF}, avec une election par le biais de liste et de soirées pour les départager.
Alexis BOULIGAND, président de l'\gls{AE}, introduit le sujet et son discours est complété par des listes d'informations impotentes, ce qui permet d'alléger son discours tout en transmettant le maximum d'informations.

Pour résumer, le fonctionnement du \gls{PDF}, sera le suivant :
\begin{itemize}
    \item Création de trois listes de 5 à 10 personnes (hors TC01)
    \item Inscription par courriel jusqu'au 25 septembre
    \item Chaque liste a une semaine pour convaincre les autres étudiant·e·s
    \item En fin des trois semaines la liste est élue par les membres de l'\gls{AE}
    \item La liste victorieuse aura la responsabilité d'organiser les différentes soirées du semestre
\end{itemize}

Le Reel peut se retrouver sur Instagram en \href{https://www.instagram.com/reel/CxQtgEXMqon/?utm_source=ig_web_copy_link&igshid=MzRlODBiNWFlZA==}{cliquant ici}.
Le post pour détailler les élections du \gls{PDF} sont à retrouver en annexe page : \pageref{subsec:interface-instagram}.

Nous pouvons voir en commentaire de ce Reel un réel retour positif des membres de l'association qui saluent le fait que l'\gls{AE} essaie de se renouveler.

\subsubsection{Soirée Char d'Assos}

Avant de commencer à parler de la réalisation de ce Reel, j'aimerais exposer la genèse de cette soirée.
Le dernier semestre (P23) j'ai eu la chance d'occuper le poste de Responsable des Relations Extérieures de l'\gls{AE}, en discutant avec son Président Alexis BOULIGAND, nous avons eu l'idée de contacter la mairie pour se présenter et voir les actions que nous pouvons réaliser conjointement.
Dans un premier temps, nous avons été redirigé vers le Bureau Information Jeunesse, ou l'où nous avons appris que Bienvenu Aux Étudiants de l'Université de Bourgogne Franche-Comté était dédié uniquement aux étudiant·e·s de la Faculté (dont nous faisons partie en temps normal).
Pour palier à ce manque d'activités et un budget non dépensé par le BIJ de Belfort, avec Alexis, nous avons décidé d'organiser une soirée inter-bureau étudiant·e·s soutenu par le BIJ et la Mairie de Belfort.
C'est de ce constat qu'a été organisée cette soirée par des étudiant·e·s Belfortain·ne·s.

Ensuite, le non Char d'Assos et une référence au Char du Lieutenant Martin mort pour la libération de Belfort durant la Seconde Guerre Mondiale, qui est exposé à côté de la Citadelle, lieu de ce festival de musique.
Ce lieu a déjà été utilisé par le Festiv'UT en 2021.

Dans ce Reel, j'ai décidé de mettre en scène trois personnes de l'\gls{UTBM} : Samuel LESNE, Benjamin DUMESNIL, Elliot DUVAL.
Tous étant impliqué dans la réalisation de ce festival.

De manière chronologique, Samuel présente de manière très générale le festival en communiquant les informations les plus importantes, ce qui permet de les transmettre en 20 secondes, ainsi si l'utilisateur·trice décidé de changer de Reel rapidement les informations les plus importantes sont communiquées :
\begin{itemize}
    \item Citadelle de Belfort
    \item Festival de musique
    \item Le 28 septembre 2023
    \item Foods trucks
    \item Deux scènes
\end{itemize}

Ensuite, Elliot continue en donnant plus amples informations sur les moyens de se désaltérer, et sur les différents organisateur·trice·s de la soirée.

Le Reel se termine par Benjamin qui invite les auditeur·trice·s à les suivre sur Instagram (\href{https://www.instagram.com/char.dassos/}{@char.dassos}) pour avoir toutes les informations en avant-première.

Durant la réalisation de ce Reel, plusieurs problèmes sont apparus, le micro qui capte l'ensemble des bruits de pas dans les graviers, l'épaisseur du char qui empêche une bonne communication entre le micro et le récepteur malencontreusement placé derrière celui-ci.

Outre ces aspects techniques, j'ai du utilisé la charte graphique de la soirée, que vous pouvez retrouver en annexe : page \pageref{subsec:charte-char-dassos}.

L'utilisation d'une musique type rock est une volonté des organisateurs, car elle représente assez bien le style de la soirée.

La vidéo étant sortie le jour même de l'événement, elle n'a peut-être pas eu l'impacte souhaitée, mais nous pouvons la retrouver ur l'Instagram de \href{https://www.instagram.com/char.dassos/}{@char.dassos} sur le lien suivant : \href{https://www.instagram.com/reel/Cxuj5g2MKov/?utm_source=ig_web_button_share_sheet&igshid=MzRlODBiNWFlZA==}{Reel Char d'Assos}.
Tout de même pour compte avec seulement 184 followers au 2/10/2023 le Reel comptabilise 66 likes ($\sfrac{1}{3}$ du public potentiel) ce qui montre encore une fois le taux de conversion et l'impact positif des Reels.

\subsubsection{Réalisation d'un questionnaire pour comprendre les attentes des cotisant·e·s}

Pour axer au mieux les vidéos sur les lieux de vie, j'ai décidé de réaliser un court questionnaire GForm, où les membres de l'association peuvent nous demander des améliorations et des retours sur ces lieux qui leur sont destinés.

Ainsi, j'ai réalisé une courte story animer pour la partager sur Instagram, malgré ses 650 vus, j'ai récolté que 3 réponses.
De ce fait, j'ai décidé de la partager sur le Discord de l'\gls{AE} pour maximiser le nombre de réponses.
Les deux hypothéses qui se dégage de l'impopularité du GForm sont :
\begin{itemize}
    \item Les membres ne réagissent pas et ne souhaitent pas s'exprimer sur des sujets qu'il·elle·s trouvent non polémique (contrairement à la blouse où nous avons eu énormément plus de retour)
    \item La story était perdu au milieu d'une communication intensive du à la fin de l'intégration
\end{itemize}

Voici un extrait des retours du GForm :
\begin{itemize}
    \item Mettre des fruits sur les lieux de vie
    \item Changer le café qui serait de moins en moins bon
\end{itemize}


Le questionnaire est consultable sur \href{https://docs.google.com/forms/d/e/1FAIpQLSfOkOUDseCfWcLwP2uz_amd-i2v_5OucU92uZAUewR6VN_P_A/viewform?usp=sf_link}{GForm}.

\subsubsection{Présentation des lieux de vie de Montbéliard}

L'\gls{AE} n'est pas présente que durant les soirées, mais tout au long de la vie des étudiant·e·s en leurs proposant divers avantages, tel que des points de restauration, des moments de partage\ldots\
C'est pour cela que je veux dédier trois Reels pour les présenter plus en détail, bien que faisant partie du bureau restreint de la \gls{AE}\footnote{Présentation des deux bureaux l'\gls{AE}, \href{https://www.instagram.com/reel/CeT9t0uAxrS/?utm_source=ig_web_copy_link&igshid=MzRlODBiNWFlZA==}{vidéo} réalisé par Samuel FOUREL et Theo LACASSIN dans le cadre de l'UV AV00.} durant mon semestre P23, et membre de l'organisation de l'Intégration 2023, je ne connaissais pas tous les avantages que cette association propose.

J'aimerais revenir sur un détail que nous retrouvons dans l'ensemble des Reels que j'ai pu tourner avec des membres actifs de l'association : le surnom\footnote{Une explication de du choix des surnoms est présentée par Florian CLOAREC, responsable Intégration 2023 dans \href{https://www.instagram.com/reel/Cws1eRdr-wV/?utm_source=ig_web_copy_link&igshid=MzRlODBiNWFlZA==}{Reel} réalisé dans le cadre de l'UV AV00.}, Alexis BOULIGAND porte le surnom Dhi faisant référence à Mahatma GANDHI et Enora SOURGET a comme surnom 27 ce qui est en lien avec Jet 27.

Revenons à la réalisation de cette vidéo.
Une volonté d'Enora est de mettre un grille-pain dans chacun des plans de ce Reel, si nous pouvons dire cet object du quotidien fait partie intégrante de la formation des EDIMs qui doivent tou·te·s réalisé un grille-pain en bakelite au cours de l'une de leur UV.
L'utilisation de ce symbole permet de mettre en confiance notre auditoire et ainsi se rapprocher de lui.

Cette vidéo s'articule autour de cinq axes :
\begin{enumerate}
    \item Présentation générale : Énora se présente, étant une habituée des lieux, car elle occupe le poste de barwoman à la Gommette, elle est la guide parfaite pour cette découverte de Montbéliard
    \item présentation de la Gommette : Énora présente le service de restauration du lieu
    \item Présentation du Saloon : lieu où les étudiant·e·s peuvent jour à la Wii, au billard et au baby-foot, ces installations sont présentes aussi à Belfort et Sevenans
    \item Présentation du FabLab : le FabLab est une particularité de Montbéliard, ici l'\gls{AE} met à disposition de ses cotisant·e·s des imprimantes 3D et du matériel de bricolage
    \item Recrutement : le pôle de Montbéliard n'a encore aucun membre mis à par Énora, donc ce Reel est l'occasion de faire un peu de recrutement pour les élections complémentaires de l'\gls{AE}
\end{enumerate}


Si nous parlons purement technique maintenant divers problèmes sont apparu avec l'utilisation du micro-cravate ce qui nous a obligé de décalé de deux semaines les prises de vues.
Lors de la prise de son qui a servi pour le montage des Reels, j'ai été rencontré à différente contrainte, un vent important en extérieur et des bruits de marteau-piqueur.
De plus, nous avons réalisé un plan dans le hall du site de Montbéliard où l'acoustique est de mauvaise qualité et avec un fort passage.
Ceci m'a contraint à retoucher fortement le son où j'ai du réduire le bruit de fond, descendre la saturation\ldots\
Pour ce faire, j'ai utilisé le logiciel \href{https://www.audacityteam.org}{Audacity}\footnote{Logiciel Open Source sous licence : GNU General Public License, Version 3.}.
Ce travail du son permet de le rendre plus audible et agréable que le son brut.

Après deux jours en ligne le Reel comporte 2650 vus et 125 likes avec 7 commentaires.
Nous pouvons dire que ce Reel est un vrai succès, car un étudiant a appris qu'il y avait un club de musique sur Montbéliard, ainsi l'objectif premier de ces courtes vidéos est réalisé.

La vidéo est sur \href{https://www.instagram.com/reel/CyEDJKTspWL/?utm_source=ig_web_copy_link&igshid=MzRlODBiNWFlZA==}{Instagram}.



\subsubsection{Présentation des lieux de vie de Sevenans}

Le lieu de vie surement le plus dynamique de l'\gls{AE}, car il occupe un lieu stratégique à Sevenans étant sous le RU c'est un lieu de passage pour un grand nombre d'étudiant·e·s.
Il regroupe un grand nombre de tables permettant de manger le midi, discuter avec ses amis, jouer à des jeux de société et à travailler tous·tes ensembles et ainsi s'entraider sur des sujets épineux.
De plus, il regroupe des tables de ping pong, des babyfoots, des billards\ldots, ce qui permet de créer une forte cohésion sur le campus de Sevennas ce qui peut disparaitre au Foyer de Belfort du fait de sa position très excentrée.
Ce lieu de vie porte le nom de MDE : la Maison Des Étudiant·e·s, ne serait mieux porter son nom pour certain·e·s adhérant·e·s de l'\gls{AE}.

Pour réaliser cette vidéo, j'ai décidé de filmer les deux responsables de cette entre : Albane WIBER ($O_2$) et Albin CHAMBRELAN (Kulh).
De plus, je voulais que cet échange soit le plus naturel possible donc je me suis limité qu'à quelques informations très simples, sans les contraindres ce qui a pu parfois donner des présentations un peu plus détails des activités et avantage de ce lieu.
Ainsi, j'ai été obligé de raccourcir bon nombre de passages dans la vidéo final.
Cette vidéo se compose de 3 axes :
\begin{enumerate}
    \item Présentation générale du lieu
    \item Présentation des activités et comment en profiter
    \item Présentation du bar
\end{enumerate}

D'un point de vue technique le plus compliqué a gérer est le niveau sonore qui entoure ces prises vidéos.

Ce Reel a réalisé 3509 vues au 13 novembre 2023 soit moins de 24 heures après sa sortie.
La vidéo est sur \href{https://www.instagram.com/reel/CzjfX8Xs5X1/?utm_source=ig_web_copy_link&igshid=MzRlODBiNWFlZA==}{Instagram}.


\subsubsection{Présentation des lieux de vie de Belfort}

Pour finir cette réalisation audio-visuelle, j'ai décidé de me consacrer à la présentation du site de Belfort avec la participation de Mathieu Pequignot responsable du site de Belfort.

Nous avons décidé de se concentrer sur la présentation du Foyer lieu emblématique de la vie étudiante de l'\gls{UTBM}, ce lieu renferme un bar ou les étudiant·e·s peuvent consommer des sodas, des bières et des snackes.
Outre, les snackes le Foyer propose une restauration rapide avec des plats surgelées.
De plus, il y a une possibilité très peu connue qui permet de faire réchauffer ou cuisiner le midi sur Belfort quand nous ne pouvons pas rentrer chez nous, ainsi nous souhaitons redynamiser ce lieu le midi.

Nous avons insisté sur le fait qu'il y a la possibilité de se mettre à l'écart durant les soirées, car depuis quelques années (post-COVID) les étudiant·e·s ont changé leur mode de consommation des soirées, de ce fait, nous essayons de communiquer sur les ensembles des actions mises en place pour améliorer le ressenti autour de la soirée.
Mathieu a aussi appuyé le fait qu'à Belfort il y a différents clubs qui propose diverses activités tel qu'un club couture, jeux de société, \ldots

Enfin, nous avons décidé de terminer sur un court point au niveau des laveries où depuis le semestre dernier un sèche-linge est mis en place.

Ce Reel a réalisé ??????? vues au ??????? novembre 2023 soit moins de 24 heures après sa sortie.
La vidéo est sur \href{https://www.instagram.com/reel/CzjfX8Xs5X1/?utm_source=ig_web_copy_link&igshid=MzRlODBiNWFlZA==}{Instagram}.


\subsection{Impacte des Reels}\label{subsec:impacte-des-reels}


Récapitulatif du nombre de vues au ?????? :

\begin{table}[h]
    \centering
    \begin{tabular}{|c|c|c|c|c|c|}
        \hline
        Compte & Nom vidéo & Vues & Likes & Commentaires & Durée \\
        \hline
        \href{https://www.instagram.com/ae_utbm/}{@ae\_utbm} & \href{https://www.instagram.com/reel/CxGShAusxDq/?utm_source=ig_web_copy_link&igshid=MzRlODBiNWFlZA==}{Présentation AE} & 2874 & 161 & 5 & 0:54 \\
        \hline
        \href{https://www.instagram.com/ae_utbm/}{@ae\_utbm} & \href{https://www.instagram.com/reel/CxQtgEXMqon/?utm_source=ig_web_copy_link&igshid=MzRlODBiNWFlZA==}{Présentation PDF} & 3004 & 101 & 2 & 0:49 \\
        \hline
        \href{https://www.instagram.com/char.dassos/}{@char.dassos} & \href{https://www.instagram.com/reel/Cxuj5g2MKov/?utm_source=ig_web_copy_link&igshid=MzRlODBiNWFlZA==}{Char d'Assos} & 2908 & 67 & 3 & 0:45 \\
        \hline
        \href{https://www.instagram.com/ae_utbm/}{@ae\_utbm} & \href{https://www.instagram.com/reel/CyEDJKTspWL/?utm_source=ig_web_copy_link&igshid=MzRlODBiNWFlZA==}{Présentation Gommette} & 3302 & 128 & 8 & 1:21 \\
        \hline
        \href{https://www.instagram.com/ae_utbm/}{@ae\_utbm} & \href{https://www.instagram.com/reel/CzjfX8Xs5X1/?utm_source=ig_web_copy_link&igshid=MzRlODBiNWFlZA==}{Présentation MDE} & 3509 & 94 & 4 & 1:26 \\
        \hline
        \href{https://www.instagram.com/ae_utbm/}{@ae\_utbm} & \href{https://www.instagram.com/}{Présentation de Belfort} & ??? & ??? & ??? & ??? \\
        \hline
        \multicolumn{2}{|c|}{Totaux :} & 000 & 000 & 000 & 00 \\
        \hline
    \end{tabular}\caption{Tableau récapitulatif des Reels}
    \label{tab:table-recap}
\end{table}
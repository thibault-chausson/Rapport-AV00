%! Author = thibaultchausson
%! Date = 23/07/2023

\subsection{Formalisation du message}

\textcolor{red}{Formalisation du message, story board}



\subsection{Les informations transmises}

\textcolor{red}{Justification des informations transmise, pourquoi le message, pourquoi ce choix technique}


\subsection{Choix de la direction artistique}\label{subsec:choix-de-la-direction-artistique}







\subsection{Réalisation des vidéos}\label{subsec:realisation-des-videos}

\subsubsection{L'\gls{AE} en Réel}\label{subsubsec:ae-en-reel}

Ce Reel a été réalisé le dimanche 10 septembre 2023, pendant une activité de l'intégration : l'après-midi jeux de société co-organiser avec le club de l'\gls{AE} le Troll Penché.
Cette activité prenait place dans le Foyer des Étudiant·e·s de Belfort.
J'ai décidé de choisir ce moment pour tourner cette vidéo, car le Foyer était dynamique et vivant, ce qui n'est pas toujours le cas, dû à son emplacement un peu excentré des bâtiments d'enseignement de l'\gls{UTBM} Belfort.
Ce dynamisme, lui a permis de se montrer sous son meilleur jour auprès des étudiant·e·s de l'\gls{UTBM}, qui parfois peuvent en avoir une image négative.
Ainsi, cette ambiance a permis de découler sur l'image de l'\gls{AE} et donc montrer une association pleine de vie.

Outre le point précédent, l'objectif premier de ce Reel est de mettre une image sur la fonction de président de l'\gls{AE}.
Durant cette courte capsule, il a pu évoquer les principales ambitions de l'association pour ce semestre :
\begin{itemize}
    \item Battle de soirées, sans dire l'objectif premier de cette bataille, qui sera évoquée dans le Reel suivant (voir partie : \ref{subsubsec:election-pdf})
    \item Présentation et mise en avant des clubs
    \item Investissement sur les lieux de vie et les sites
\end{itemize}


La vidéo se termine en demandant une implication des membres pour qu'il·elle·s nous apportent des idées sur ceux qu'il·elle·s veulent voir de nouveaux.
À l'heure où je rédige ce document, deux idées ont été évoquées :
\begin{itemize}
    \item Achat d'une console nouvelle génération
    \item Achat d'un Google Chrome Cast
\end{itemize}

Le Reel peut se retrouver sur Instagram en \href{https://www.instagram.com/reel/CxGShAusxDq/?utm_source=ig_web_copy_link&igshid=MzRlODBiNWFlZA==}{cliquant ici}.


\subsubsection{Élection du \gls{PDF}}\label{subsubsec:election-pdf}


Ce Reel a été tourné le même jour que le Reel de présentation de l'AE (voir partie \ref{subsubsec:ae-en-reel}), ainsi, nous retrouvons les mêmes motivations qu'évoqué précédement.

Ce Reel a pour but de présenter le nouveau fonctionnement du \gls{PDF}, avec une election par le biais de liste et de soirées pour les départager.
Alexis BOULIGAND, président de l'\gls{AE}, introduit le sujet et son discours est complété par des listes d'informations impotentes, ce qui permet d'alléger son discours tout en transmettant le maximum d'informations.

Pour résumer, le fonctionnement du \gls{PDF}, sera le suivant :
\begin{itemize}
    \item Création de trois listes de 5 à 10 personnes (hors TC01)
    \item Inscription par courriel jusqu'au 25 septembre
    \item Chaque liste a une semaine pour convaincre les autres étudiant·e·s
    \item En fin des trois semaines la liste est élue par les membres de l'\gls{AE}
    \item La liste victorieuse aura la responsabilité d'organiser les différentes soirées du semestre
\end{itemize}

Le Reel peut se retrouver sur Instagram en \href{https://www.instagram.com/reel/CxQtgEXMqon/?utm_source=ig_web_copy_link&igshid=MzRlODBiNWFlZA==}{cliquant ici}.

Nous pouvons voir en commentaire de ce Reel un réel retour positif des membres de l'association qui saluent le fait que l'\gls{AE} essaie de se renouveler.

\subsubsection{Soirée Char d'Assaut}


\subsection{Impacte des Reels}\label{subsec:impacte-des-reels}

\subsection{Scripts}\label{subsec:scripts}

\begin{dialogue}
    \speak{Thibault} \lipsum[1]
    \speak{Laura} \lipsum[2]
\end{dialogue}
%! Author = thibaultchausson
%! Date = 23/07/2023

\subsection{Formalisation du message}\label{subsec:formalisation-du-message}

\subsubsection{L'AE en Réel}
\begin{itemize}
    \item \underline{Image 1 :} Présentation de l'\gls{AE}, des projets du semestre et du président.
    \item \underline{Image 2 :} Battle de soirées pour l'élection du PDF, valorisation des clubs, investissement du semestre.
\end{itemize}

\subsubsection{Élection du PDF}
\begin{itemize}
    \item \underline{Image 1 :} Président de l'AE, Alexis BOULIGAND, introduisant le nouveau fonctionnement du \gls{PDF}.
    \item \underline{Image 2 :} Informations importantes sur les dates et le fonctionnement de la battle de soirées.
    \item \underline{Image 3 :} Informations sur la manière de s'inscrire.
    \item \underline{Image 4 :} Composition des équipes et organisation.
    \item \underline{Image 5 :} Date limite pour les candidatures.
\end{itemize}

\subsubsection{Soirée Char d’Assos}
\begin{itemize}
    \item \underline{Image 1 :} Samuel présente le concept du festival.
    \item \underline{Image 2 :} Présentation des aménagements sur place, restauration, bar, deux scènes.
    \item \underline{Image 3 :} Les organisateurs : BDE de Belfort et Montbéliard.
    \item \underline{Image 4 :} Date et heure de début.
\end{itemize}

\subsubsection{Présentation des lieux de vie de Montbéliard}
\begin{itemize}
    \item \underline{Image 1 :} Présentation générale des lieux par Énora.
    \item \underline{Image 2 :} La Gommette, le lieu de vie.
    \item \underline{Image 3 :} Le saloon, espace de détente.
    \item \underline{Image 4 :} Le fablab.
    \item \underline{Image 5 :} Recrutement pour le Pôle de Montbéliard.
\end{itemize}

\subsubsection{Présentation des lieux de vie de Sevenans}
\begin{itemize}
    \item \underline{Image 1 :} Albin présente la MDE.
    \item \underline{Image 2 :} Présentation des activités de la MDE.
    \item \underline{Image 3 :} Présentation du bar de la MDE.
\end{itemize}

\subsubsection{Présentation des lieux de vie de Belfort}
\begin{itemize}
    \item \underline{Image 1 :} Présentation générale par Mathieu.
    \item \underline{Image 2 :} Présentation du Foyer et de ses articles en vente.
    \item \underline{Image 3 :} Présentation de la cuisine pour réchauffer son repas du midi.
    \item \underline{Image 4 :} Présentation de l'espace détente de Belfort.
    \item \underline{Image 5 :} Présentation de la laverie.
    \item \underline{Image 6 :} Fin.
\end{itemize}

\subsection{Les informations transmises}\label{subsec:les-informations-transmises}

Comme vu précédemment, j'ai choisi d'utiliser le format Reel d'Instagram en raison de l'efficacité de ce réseau social et de ce nouveau format.
Ainsi, je pourrai toucher un grand nombre d'étudiant·e·s de l'\gls{UTBM} par ce moyen.

Le message principal transmis dans l'ensemble des vidéos est que la vie associative de l'\gls{UTBM} est très active, notamment sur ses trois lieux iconiques, parfois méconnus de certain·e·s étudiant·e·s, en raison de la répartition géographique de l'\gls{UTBM} sur trois sites distants.
En présentant les activités à Belfort, Montbéliard et Sevenans, j'espère encourager davantage de personnes à fréquenter ces lieux pour partager des moments de vie inoubliables avec leurs ami·e·s.
Nous espérons également à l'\gls{AE} que la mise en avant des responsables des lieux pourra motiver même les plus réfractaires à s'engager dans la vie associative de l'\gls{UTBM}.

Pour transmettre ce message, j'ai opté pour l'utilisation d'un smartphone pour filmer ces vidéos.
Instagram compresse les vidéos, rendant inutile l'utilisation du matériel professionnel dont dispose l'\gls{AE} pour ce type de vidéo.
J'ai choisi d'utiliser des micros-cravates pour toutes les prises de son, car à distance, le son capté par le micro de l'iPhone est de qualité insuffisante, et un son de mauvaise qualité pourrait nuire à la communication des informations.


\subsection{Choix de la direction artistique}\label{subsec:choix-de-la-direction-artistique}

Pour réaliser ces vidéos, j'ai opté pour l'utilisation du logiciel Canva, permettant de créer facilement un montage dynamique et captivant.
Concernant les couleurs de l'arrière-plan, j'ai choisi de conserver celles du logo de l'\gls{AE} pour maintenir une cohérence visuelle.

Pour les différentes vidéos destinées à l'\gls{AE}, j'ai décidé de conserver le même style graphique afin d'assurer une unité entre les différents Reels.
Toutefois, le support a été modifié pour la soirée Char d'Asso, car cet événement était une collaboration inter-BDE.


\subsection{Réalisation des vidéos}\label{subsec:realisation-des-videos}

\subsubsection{L'\gls{AE} en Réel}\label{subsubsec:ae-en-reel}

Ce Reel a été réalisé le dimanche 10 septembre 2023, pendant une activité de l'intégration : l'après-midi jeux de société co-organisé avec le club de l'\gls{AE}, le Troll Penché.
Cette activité se déroulait dans le Foyer des Étudiant·e·s de Belfort.
J'ai choisi ce moment pour tourner la vidéo, car le Foyer était particulièrement dynamique et vivant, ce qui n'est pas toujours le cas en raison de son emplacement un peu excentré par rapport aux bâtiments d'enseignement de l'\gls{UTBM} à Belfort.
Ce dynamisme a offert l'opportunité de présenter le Foyer sous son meilleur jour auprès des étudiant·e·s de l'\gls{UTBM}, qui peuvent parfois en avoir une image négative.
Cette ambiance a également contribué à refléter l'image vivante de l'\gls{AE}.

En outre, l'objectif principal de ce Reel était de mettre en lumière la fonction de président de l'\gls{AE}.
Au cours de cette courte vidéo, le président a pu évoquer les principales ambitions de l'association pour ce semestre :
\begin{itemize}
    \item Organiser une battle de soirées, sans révéler l'objectif premier de cette bataille, qui sera évoqué dans le Reel suivant (voir partie : \ref{subsubsec:election-pdf}).
    \item Présentation et mise en avant des clubs.
    \item Investissement dans les lieux de vie et les sites.
\end{itemize}

La vidéo se conclut en invitant les membres à partager leurs idées sur ce qu'il·elle·s aimeraient voir de nouveau.
À l'heure où je rédige ce document, deux suggestions ont été reçues :
\begin{itemize}
    \item Achat d'une console de nouvelle génération.
    \item Achat d'un Google Chrome Cast.
\end{itemize}

Le Reel peut se retrouver sur Instagram en \href{https://www.instagram.com/reel/CxGShAusxDq/?utm_source=ig_web_copy_link&igshid=MzRlODBiNWFlZA==}{cliquant ici}.


\subsubsection{Élection du \gls{PDF}}\label{subsubsec:election-pdf}


Ce Reel a été tourné le même jour que le Reel de présentation de l'AE (voir partie \ref{subsubsec:ae-en-reel}), ainsi, nous retrouvons les mêmes motivations qu'évoqué précédemment.

Ce Reel a pour but de présenter le nouveau fonctionnement du \gls{PDF}, avec une élection par le biais de liste et de soirées pour les départager.
Alexis BOULIGAND, président de l'\gls{AE}, introduit le sujet et son discours est complété par des listes d'informations importantes, ce qui permet d'alléger son discours tout en transmettant le maximum d'informations.

Pour résumer, le fonctionnement du \gls{PDF}, sera le suivant :
\begin{itemize}
    \item Création de trois listes de 5 à 10 personnes (hors TC01)
    \item Inscription par courriel jusqu'au 25 septembre
    \item Chaque liste a une semaine pour convaincre les autres étudiant·e·s
    \item À fin des trois semaines la liste est élue par les membres de l'\gls{AE}
    \item La liste victorieuse aura la responsabilité d'organiser les différentes soirées du semestre
\end{itemize}

Le Reel peut se retrouver sur Instagram en \href{https://www.instagram.com/reel/CxQtgEXMqon/?utm_source=ig_web_copy_link&igshid=MzRlODBiNWFlZA==}{cliquant ici}.
Le post pour détailler les élections du \gls{PDF} sont à retrouver en annexe page : \pageref{subsec:interface-instagram}.

Nous pouvons voir en commentaire de ce Reel un réel retour positif des membres de l'association qui saluent le fait que l'\gls{AE} essaie de se renouveler.

\subsubsection{Soirée Char d'Assos}

Avant de commencer à parler de la réalisation de ce Reel, j'aimerais exposer la genèse de cette soirée.
Le semestre dernier (P23) j'ai eu la chance d'occuper le poste de Responsable des Relations Extérieures de l'\gls{AE}, en discutant avec son Président Alexis BOULIGAND, nous avons eu l'idée de contacter la mairie pour se présenter et voir les actions que nous pouvons réaliser conjointement.
Dans un premier temps, nous avons été redirigé vers le Bureau Information Jeunesse, ou nous avons appris que Bienvenu Aux Étudiants de l'Université de Bourgogne Franche-Comté était dédié uniquement aux étudiant·e·s de la Faculté (dont nous faisons partie en temps normal).
Pour palier à ce manque d'activités et un budget non dépensé par le BIJ de Belfort, avec Alexis, nous avons décidé d'organiser une soirée inter-bureau étudiant·e·s soutenu par le BIJ et la Mairie de Belfort.
C'est de ce constat qu'a été organisée cette soirée par des étudiant·e·s Belfortain·ne·s.

Ensuite, le non Char d'Assos et une référence au Char du Lieutenant Martin mort pour la libération de Belfort durant la Seconde Guerre Mondiale, qui est exposé à côté de la Citadelle, lieu de ce festival de musique.
Ce lieu a déjà été utilisé par le Festiv'UT en 2021.

Dans ce Reel, j'ai décidé de mettre en scène trois personnes de l'\gls{UTBM} : Samuel LESNE, Benjamin DUMESNIL, Elliot DUVAL.
Tous étant impliqué dans la réalisation de ce festival.

De manière chronologique, Samuel présente dans les grandes lignes ce festival en communiquant les informations les plus importantes, ce qui permet de les transmettre en 20 secondes, ainsi si l'utilisateur·trice décidé de changer de Reel rapidement les informations les plus importantes sont communiquées :
\begin{itemize}
    \item Citadelle de Belfort
    \item Festival de musique
    \item Le 28 septembre 2023
    \item Foods trucks
    \item Deux scènes
\end{itemize}

Ensuite, Elliot continue en donnant plus amples informations sur les moyens de se désaltérer, et sur les différent·e·s organisateur·trice·s de la soirée.

Le Reel se termine par Benjamin qui invite les auditeur·trice·s à les suivre sur Instagram (\href{https://www.instagram.com/char.dassos/}{@char.dassos}) pour avoir toutes les informations en avant-première.

Durant la réalisation de ce Reel, plusieurs problèmes sont apparus, le micro qui capte l'ensemble des bruits de pas dans les graviers, l'épaisseur du char qui empêche une bonne communication entre le micro et le récepteur malencontreusement placé derrière celui-ci.

Outre ces aspects techniques, j'ai du utilisé la charte graphique de la soirée, que vous pouvez retrouver en annexe : page \pageref{subsec:charte-char-dassos}.

L'utilisation d'une musique type rock est une volonté des organisateur·trice·s, car elle représente assez bien le style de la soirée.

La vidéo étant sortie le jour même de l'événement, elle n'a peut-être pas eu l'impacte souhaitée, mais nous pouvons la retrouver sur l'Instagram de \href{https://www.instagram.com/char.dassos/}{@char.dassos} sur le lien suivant : \href{https://www.instagram.com/reel/Cxuj5g2MKov/?utm_source=ig_web_button_share_sheet&igshid=MzRlODBiNWFlZA==}{Reel Char d'Assos}.
Tout de même pour compte avec seulement 184 followers au 2 octobre 2023 le Reel comptabilise 66 likes ($\sfrac{1}{3}$ du public potentiel) ce qui montre encore une fois le taux de conversion et l'impact positif des Reels.

\subsubsection{Réalisation d'un questionnaire pour comprendre les attentes des cotisant·e·s}

Pour mieux cibler les vidéos sur les lieux de vie, j'ai créé un court questionnaire via Google Forms, permettant aux membres de l'association de suggérer des améliorations et de donner leur avis sur ces espaces qui leur sont destinés.

J'ai ensuite réalisé une courte story animée pour la partager sur Instagram.
Malgré ses 650 vues, le questionnaire n'a recueilli que 3 réponses.
De ce fait, j'ai décidé de la partager également sur le Discord de l'\gls{AE} afin de maximiser le nombre de retours.
Deux hypothèses peuvent expliquer le faible engouement pour ce Google Forms :
\begin{itemize}
    \item Les membres ne réagissent pas et ne souhaitent pas s'exprimer sur des sujets qu'il·elle·s considèrent comme non polémiques (contrairement à la blouse, sujet sur lequel nous avons eu nettement plus de retours).
    \item La story était noyée dans une communication intensive liée à la fin de l'intégration.
\end{itemize}

Voici un extrait des retours obtenus via le Google Forms :
\begin{itemize}
    \item Proposition de mettre des fruits à disposition dans les lieux de vie.
    \item Suggestion de changer le café, jugé de moins en moins bon.
\end{itemize}


Le questionnaire est consultable sur \href{https://docs.google.com/forms/d/e/1FAIpQLSfOkOUDseCfWcLwP2uz_amd-i2v_5OucU92uZAUewR6VN_P_A/viewform?usp=sf_link}{GForm}.

\subsubsection{Présentation des lieux de vie de Montbéliard}\label{subsubsec:montbeliard}

L'\gls{AE} ne se limite pas aux soirées, mais joue un rôle tout au long de la vie des étudiant·e·s en leur proposant divers avantages, tels que des points de restauration et des moments de partage.
C'est pourquoi j'ai décidé de dédier trois Reels pour les présenter plus en détail.
Bien que faisant partie du bureau restreint de l'\gls{AE}\footnote{Présentation des deux bureaux de l'\gls{AE}, \href{https://www.instagram.com/reel/CeT9t0uAxrS/?utm_source=ig_web_copy_link&igshid=MzRlODBiNWFlZA==}{vidéo} réalisée par Samuel FOUREL et Theo LACASSIN dans le cadre de l'UV AV00.} durant mon semestre P23, et membre de l'organisation de l'Intégration 2023, je ne connaissais pas tous les avantages proposés par cette association.

J'aimerais revenir sur un détail présent dans tous les Reels que j'ai tournés avec des membres actifs de l'association : le surnom\footnote{Une explication du choix des surnoms est présentée par Florian CLOAREC, responsable de l'Intégration 2023, dans \href{https://www.instagram.com/reel/Cws1eRdr-wV/?utm_source=ig_web_copy_link&igshid=MzRlODBiNWFlZA==}{ce Reel} réalisé dans le cadre de l'UV AV00.}.
Par exemple, Alexis BOULIGAND porte le surnom Dhi, en référence à Mahatma GANDHI, et Enora SOURGET a pour surnom 27, en lien avec le Jet 27.

Revenons à la réalisation de cette vidéo.
Enora souhaite inclure un grille-pain dans chaque plan du Reel.
Cet objet du quotidien fait partie intégrante de la formation des EDIMs, qui doivent tous réaliser un grille-pain en bakélite au cours de l'une de leurs UV. L'utilisation de ce symbole vise à mettre en confiance notre audience et à se rapprocher d'elle.

La vidéo s'articule autour de cinq axes :
\begin{enumerate}
    \item Présentation générale : Énora se présente en tant qu'habituée des lieux et barwoman à la Gommette, elle est la guide idéale pour cette découverte de Montbéliard.
    \item Présentation de la Gommette : Énora présente le service de restauration.
    \item Présentation du Saloon : lieu où les étudiant·e·s peuvent jouer à la Wii, au billard et au baby-foot, installations également présentes à Belfort et Sevenans.
    \item Présentation du FabLab : une particularité de Montbéliard où l'\gls{AE} met à disposition des imprimantes 3D et du matériel de bricolage.
    \item Recrutement : le pôle de Montbéliard n'ayant aucun membre à part Énora, ce Reel est l'occasion de recruter pour les élections complémentaires de l'\gls{AE}.
\end{enumerate}

Sur le plan technique, divers problèmes sont survenus avec l'utilisation du micro-cravate, ce qui a retardé les prises de vue de deux semaines.
Lors de la prise de son pour le montage des Reels, j'ai rencontré différentes contraintes, telles qu'un vent important en extérieur et des bruits de marteau-piqueur.
De plus, un plan a été réalisé dans le hall du site de Montbéliard, où l'acoustique est mauvaise et avec un passage intense d'étudiant·e·s.
J'ai dû retoucher fortement le son, réduire le bruit de fond, diminuer la saturation, etc., en utilisant le logiciel \href{https://www.audacityteam.org}{Audacity}\footnote{Logiciel Open Source sous licence : GNU General Public License, Version 3.}.
Ce travail a permis de rendre le son plus audible et agréable que le son brut.

Après deux jours en ligne, le Reel comptabilise 2650 vues et 125 likes avec 7 commentaires.
Ce Reel est un vrai succès, car un étudiant a découvert l'existence d'un club de musique à Montbéliard, réalisant ainsi l'objectif premier de ces courtes vidéos.

La vidéo est disponible sur \href{https://www.instagram.com/reel/CyEDJKTspWL/?utm_source=ig_web_copy_link&igshid=MzRlODBiNWFlZA==}{Instagram}.


\subsubsection{Présentation des lieux de vie de Sevenans}\label{subsubsec:MDE}

Le lieu de vie le plus dynamique de l'\gls{AE}, en raison de sa position stratégique à Sevenans sous le RU, est un point de passage pour un grand nombre d'étudiant·e·s.
Ce lieu, nommé MDE : la Maison Des Étudiant·e·s, offre un espace où l'on peut manger le midi, discuter avec des ami·e·s, jouer à des jeux de société ou travailler ensemble.
Avec ses tables de ping-pong, baby-foot, billards, etc., il contribue à créer une forte cohésion sur le campus de Sevenans, un aspect parfois moins présent au Foyer de Belfort en raison de son emplacement excentré.

Pour réaliser cette vidéo, j'ai choisi de filmer les deux responsables de cet espace : Albane WIBER ($O_2$) et Albin CHAMBRELAN (Kulh).
Souhaitant capturer un échange naturel, je me suis limité à quelques informations simples sans les contraindre, ce qui a parfois conduit à des présentations plus détaillées des activités et avantages de ce lieu.
J'ai donc dû raccourcir certains passages dans la version définitive de la vidéo.
Elle se compose de trois axes :
\begin{enumerate}
    \item Présentation générale du lieu.
    \item Présentation des activités et comment en profiter.
    \item Présentation du bar.
\end{enumerate}

D'un point de vue technique, le plus grand défi a été de gérer le niveau sonore durant les prises de vue.

Ce Reel a atteint 3509 vues au 13 novembre 2023, soit moins de 24 heures après sa publication.
La vidéo est disponible sur \href{https://www.instagram.com/reel/CzjfX8Xs5X1/?utm_source=ig_web_copy_link&igshid=MzRlODBiNWFlZA==}{Instagram}.

\subsubsection{Présentation des lieux de vie de Belfort}

Pour clore cette réalisation audiovisuelle, j'ai choisi de me consacrer à la présentation du site de Belfort, avec la participation de Mathieu PEQUIGNOT, responsable du site.

Nous avons décidé de nous concentrer sur la présentation du Foyer, un lieu emblématique de la vie étudiante de l'\gls{UTBM}.
Ce lieu abrite un bar où les étudiant·e·s peuvent consommer des sodas, des bières et des snacks.
Outre les snacks, le Foyer propose également une restauration rapide à base de plats surgelés.
De plus, une possibilité peu connue permet de réchauffer ou de cuisiner sur place le midi à Belfort, pour celles et ceux qui ne peuvent pas rentrer chez eux.
Ainsi, nous souhaitons redynamiser ce lieu pendant la pause déjeuner.

Nous avons également insisté sur la possibilité de s'isoler durant les soirées, car depuis quelques années (notamment après la pandémie de COVID-19), les habitudes de consommation des soirées étudiantes ont évolué.
De ce fait, nous tentons de communiquer sur l'ensemble des actions mises en place pour améliorer le ressenti autour des soirées.
Mathieu a aussi souligné l'existence de différents clubs à Belfort proposant diverses activités, tels qu'un club de couture, de jeux de société, etc.

Pour conclure, nous avons évoqué la laverie, mentionnant l'installation récente d'un sèche-linge depuis le semestre dernier.

Ce Reel a réalisé ??????? vues au ??????? novembre 2023 soit moins de 24 heures après sa sortie.
La vidéo est sur \href{https://www.instagram.com/reel/CzjfX8Xs5X1/?utm_source=ig_web_copy_link&igshid=MzRlODBiNWFlZA==}{Instagram}.


\subsection{Impacte des Reels}\label{subsec:impacte-des-reels}


Récapitulatif du nombre de vues au ?????? :

\begin{table}[h]
    \centering
    \begin{tabular}{|c|c|c|c|c|c|}
        \hline
        Compte & Nom vidéo & Vues & Likes & Commentaires & Durée \\
        \hline
        \href{https://www.instagram.com/ae_utbm/}{@ae\_utbm} & \href{https://www.instagram.com/reel/CxGShAusxDq/?utm_source=ig_web_copy_link&igshid=MzRlODBiNWFlZA==}{Présentation AE} & 2874 & 161 & 5 & 0:54 \\
        \hline
        \href{https://www.instagram.com/ae_utbm/}{@ae\_utbm} & \href{https://www.instagram.com/reel/CxQtgEXMqon/?utm_source=ig_web_copy_link&igshid=MzRlODBiNWFlZA==}{Présentation PDF} & 3004 & 101 & 2 & 0:49 \\
        \hline
        \href{https://www.instagram.com/char.dassos/}{@char.dassos} & \href{https://www.instagram.com/reel/Cxuj5g2MKov/?utm_source=ig_web_copy_link&igshid=MzRlODBiNWFlZA==}{Char d'Assos} & 2908 & 67 & 3 & 0:45 \\
        \hline
        \href{https://www.instagram.com/ae_utbm/}{@ae\_utbm} & \href{https://www.instagram.com/reel/CyEDJKTspWL/?utm_source=ig_web_copy_link&igshid=MzRlODBiNWFlZA==}{Présentation Gommette} & 3302 & 128 & 8 & 1:21 \\
        \hline
        \href{https://www.instagram.com/ae_utbm/}{@ae\_utbm} & \href{https://www.instagram.com/reel/CzjfX8Xs5X1/?utm_source=ig_web_copy_link&igshid=MzRlODBiNWFlZA==}{Présentation MDE} & 3509 & 94 & 4 & 1:26 \\
        \hline
        \href{https://www.instagram.com/ae_utbm/}{@ae\_utbm} & \href{https://www.instagram.com/}{Présentation de Belfort} & ??? & ??? & ??? & ??? \\
        \hline
        \multicolumn{2}{|c|}{Totaux :} & 000 & 000 & 000 & 00 \\
        \hline
    \end{tabular}\caption{Tableau récapitulatif des Reels}
    \label{tab:table-recap}
\end{table}



\subsection{Choix des Reels présentés}\label{subsec:choix-des-reels-presentes}

Pour conclure ce projet, j'ai choisi de présenter les trois vidéos suivantes, pour une durée totale de 3 minutes et 42 secondes :

\begin{itemize}
    \item Présentation de l'\gls{AE} (voir partie : \ref{subsubsec:ae-en-reel}) : cette vidéo a été sélectionnée, car elle offre une vision générale de l'association à l'ensemble des cotisant·e·s.
    Elle met en avant les objectifs du président pour le semestre, ainsi que la dynamique du Foyer en ce début de semestre.
    De plus, elle inaugure le format de l'\gls{AE} en Réel.
    \item Présentation de Montbéliard (voir partie : \ref{subsubsec:montbeliard}) : cette vidéo a été retenue pour sa capacité à montrer la dynamique du lieu de vie de Montbéliard, souvent méconnu des autres étudiant·e·s de l'\gls{UTBM}.
    Personnellement, j'apprécie cette vidéo, car elle m'a permis de découvrir plus en détail les initiatives de l'\gls{AE} à Montbéliard.
    \item Présentation de la MDE (voir partie : \ref{subsubsec:MDE}) : pour terminer cette sélection, j'ai choisi la présentation de la MDE. Ce lieu de vie est essentiel, car il permet aux nouveaux arrivant·e·s à l'\gls{UTBM} de faire des rencontres humainement enrichissantes.
\end{itemize}

J'espère que le visionnage de ces courtes vidéos sera aussi passionnant pour vous que leur création l'a été pour moi.
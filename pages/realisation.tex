%! Author = thibaultchausson
%! Date = 23/07/2023

\subsection{Formalisation du message}

\textcolor{red}{Formalisation du message, story board}



\subsection{Les informations transmises}

\textcolor{red}{Justification des informations transmise, pourquoi le message, pourquoi ce choix technique}


\subsection{Choix de la direction artistique}\label{subsec:choix-de-la-direction-artistique}







\subsection{Réalisation des vidéos}\label{subsec:realisation-des-videos}

\subsubsection{L'\gls{AE} en Réel}\label{subsubsec:ae-en-reel}

Ce Reel a été réalisé le dimanche 10 septembre 2023, pendant une activité de l'intégration : l'après-midi jeux de société co-organiser avec le club de l'\gls{AE} le Troll Penché.
Cette activité prenait place dans le Foyer des Étudiant·e·s de Belfort.
J'ai décidé de choisir ce moment pour tourner cette vidéo, car le Foyer était dynamique et vivant, ce qui n'est pas toujours le cas, dû à son emplacement un peu excentré des bâtiments d'enseignement de l'\gls{UTBM} Belfort.
Ce dynamisme, lui a permis de se montrer sous son meilleur jour auprès des étudiant·e·s de l'\gls{UTBM}, qui parfois peuvent en avoir une image négative.
Ainsi, cette ambiance a permis de découler sur l'image de l'\gls{AE} et donc montrer une association pleine de vie.

Outre le point précédent, l'objectif premier de ce Reel est de mettre une image sur la fonction de président de l'\gls{AE}.
Durant cette courte capsule, il a pu évoquer les principales ambitions de l'association pour ce semestre :
\begin{itemize}
    \item Battle de soirées, sans dire l'objectif premier de cette bataille, qui sera évoquée dans le Reel suivant (voir partie : \ref{subsubsec:election-pdf})
    \item Présentation et mise en avant des clubs
    \item Investissement sur les lieux de vie et les sites
\end{itemize}


La vidéo se termine en demandant une implication des membres pour qu'il·elle·s nous apportent des idées sur ceux qu'il·elle·s veulent voir de nouveaux.
À l'heure où je rédige ce document, deux idées ont été évoquées :
\begin{itemize}
    \item Achat d'une console nouvelle génération
    \item Achat d'un Google Chrome Cast
\end{itemize}

Le Reel peut se retrouver sur Instagram en \href{https://www.instagram.com/reel/CxGShAusxDq/?utm_source=ig_web_copy_link&igshid=MzRlODBiNWFlZA==}{cliquant ici}.


\subsubsection{Élection du \gls{PDF}}\label{subsubsec:election-pdf}


Ce Reel a été tourné le même jour que le Reel de présentation de l'AE (voir partie \ref{subsubsec:ae-en-reel}), ainsi, nous retrouvons les mêmes motivations qu'évoqué précédement.

Ce Reel a pour but de présenter le nouveau fonctionnement du \gls{PDF}, avec une election par le biais de liste et de soirées pour les départager.
Alexis BOULIGAND, président de l'\gls{AE}, introduit le sujet et son discours est complété par des listes d'informations impotentes, ce qui permet d'alléger son discours tout en transmettant le maximum d'informations.

Pour résumer, le fonctionnement du \gls{PDF}, sera le suivant :
\begin{itemize}
    \item Création de trois listes de 5 à 10 personnes (hors TC01)
    \item Inscription par courriel jusqu'au 25 septembre
    \item Chaque liste a une semaine pour convaincre les autres étudiant·e·s
    \item En fin des trois semaines la liste est élue par les membres de l'\gls{AE}
    \item La liste victorieuse aura la responsabilité d'organiser les différentes soirées du semestre
\end{itemize}

Le Reel peut se retrouver sur Instagram en \href{https://www.instagram.com/reel/CxQtgEXMqon/?utm_source=ig_web_copy_link&igshid=MzRlODBiNWFlZA==}{cliquant ici}.
Le post pour détailler les élections du \gls{PDF} sont à retrouver en annexe page : \pageref{subsec:interface-instagram}.

Nous pouvons voir en commentaire de ce Reel un réel retour positif des membres de l'association qui saluent le fait que l'\gls{AE} essaie de se renouveler.

\subsubsection{Soirée Char d'Assos}

Avant de commencer à parler de la réalisation de ce Reel, j'aimerais exposer la genèse de cette soirée.
Le dernier semestre (P23) j'ai eu la chance d'occuper le poste de Responsable des Relations Extérieures de l'\gls{AE}, en discutant avec son Président Alexis BOULIGAND, nous avons eu l'idée de contacter la mairie pour se présenter et voir les actions que nous pouvons réaliser conjointement.
Dans un premier temps, nous avons été redirigé vers le Bureau Information Jeunesse, ou l'où nous avons appris que Bienvenu Aux Étudiants de l'Université de Bourgogne Franche-Comté était dédié uniquement aux étudiant·e·s de la Faculté (dont nous faisons partie en temps normal).
Pour palier à ce manque d'activités et un budget non dépensé par le BIJ de Belfort, avec Alexis, nous avons décidé d'organiser une soirée inter-bureau étudiant·e·s soutenu par le BIJ et la Mairie de Belfort.
C'est de ce constat qu'a été organisée cette soirée par des étudiant·e·s Belfortain·ne·s.

Ensuite, le non Char d'Assos et une référence au Char du Lieutenant Martin mort pour la libération de Belfort durant la Seconde Guerre Mondiale, qui est exposé à côté de la Citadelle, lieu de ce festival de musique.
Ce lieu a déjà été utilisé par le Festiv'UT en 2021.

Dans ce Reel, j'ai décidé de mettre en scène trois personnes de l'\gls{UTBM} : Samuel LESNE, Benjamin DUMESNIL, Elliot DUVAL.
Tous étant impliqué dans la réalisation de ce festival.

De manière chronologique, Samuel présente de manière très générale le festival en communiquant les informations les plus importantes, ce qui permet de les transmettre en 20 secondes, ainsi si l'utilisateur·trice décidé de changer de Reel rapidement les informations les plus importantes sont communiquées :
\begin{itemize}
    \item Citadelle de Belfort
    \item Festival de musique
    \item Le 28 septembre 2023
    \item Foods trucks
    \item Deux scènes
\end{itemize}

Ensuite, Elliot continue en donnant plus amples informations sur les moyens de se désaltérer, et sur les différents organisateur·trice·s de la soirée.

Le Reel se termine par Benjamin qui invite les auditeur·trice·s à les suivre sur Instagram (\href{https://www.instagram.com/char.dassos/}{@char.dassos}) pour avoir toutes les informations en avant-première.

Durant la réalisation de ce Reel, plusieurs problèmes sont apparus, le micro qui capte l'ensemble des bruits de pas dans les graviers, l'épaisseur du char qui empêche une bonne communication entre le micro et le récepteur malencontreusement placé derrière celui-ci.

Outre ces aspects techniques, j'ai du utilisé la charte graphique de la soirée, que vous pouvez retrouver en annexe : page \pageref{subsec:charte-char-dassos}.

L'utilisation d'une musique type rock est une volonté des organisateurs, car elle représente assez bien le style de la soirée.

\subsection{Impacte des Reels}\label{subsec:impacte-des-reels}

\subsection{Scripts}\label{subsec:scripts}

\begin{dialogue}
    \speak{Thibault} \lipsum[1]
    \speak{Laura} \lipsum[2]
\end{dialogue}
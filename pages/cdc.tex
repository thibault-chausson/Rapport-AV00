%! Author = thibaultchausson
%! Date = 23/07/2023

\subsection{Sujet}\label{subsec:sujet}

Le but de ce projet audiovisual est de réaliser de courtes vidéos appelé Reel sur Instagram, pour montrer la diversité de l'engagement associatif, mais plus particulièrement l'engagement des étudiant·e·s.

Le projet montrera divers moments de vie estudiantine, des interviews d'étudiant·e·s, pour promouvoir les activités dont il·elle·s sont responsable au sein de l'AE UTBM.

Ainsi, l'objectif principal de ce projet est de communiquer sur les différents événements de l'AE, ainsi que ses temps fort, je peux citer de manière non exhaustive, le nouveau principe d'élection pour le Pôle Des Festivités (PDF), les soirées d'ampleurs co-réaliser avec les BDE de Belfort, mais encore promouvoir l'engagement dans l'association\ldots

Comme évoqué procurement, la motivation première de ce sujet est la promotion de la vie associative estudiantine.
Je suis motivé pour promouvoir l'AE en lui donnant une autre image que celle actuelle, en montrant que derrière cette association tentaculaire, il y a des étudiant·e·s qui sont motivés pour améliorer le cadre de vie de leurs paires.
De plus, je trouve que donner la parole aux différents instigateur·trice·s de ces changements est d'autant plus gratifiant pour eux.

\textcolor{red}{Justification du sujet : objectif, motivation (en quoi la forme aide le fond) changement important dopamine}


\subsection{Méthodes}\label{subsec:methodes}


\subsubsection{Le fond}


\subsubsection{La forme}

Pour mener à bien ce projet, j'ai décidé de réaliser de courtes capsules vidéos dynamiques de 1 à 1 minute 30, ce qui permet de garder mon auditoire attentif au message que je souhaite communiquer.
De ce fait, le format de Reels Instagram\footnote{Un Reel est une vidéo courte et divertisante de moins de 90 secondes (\href{https://about.instagram.com/fr-fr/features/reels}{Reels Instagram}).} correspond parfaitement à ces contraintes.

Les plates-formes numériques de communication en réseau ont révolutionné la façon dont nous communiquons, obtenons des informations et nous divertissons ; elles ont eu un impact majeur sur les nouvelles générations.
Nous pouvons remarquer que l'utilisation des réseaux sociaux a littéralement explosée avec la pandémie de COVID-19, ce qui peut s'expliquer par les divers confinements.
L'activité des nouvelles générations sur TikTok, Instagram a modifié les comportements sociaux de la génération actuelle.
Les likes, les commentaires et les abonnements stimulent le système de récompense dopaminergique, qui est la base des comportements addictifs\cite{pedrouzo2023hyperconnected}.
Ce qui implique une rétention plus importante des utilisateur·trice·s sur ces plateformes.
Ainsi, il parait pertinent de focaliser nos efforts sur ce mode de communication pour toucher et incité le plus d'étudiant·e·s de l'UTBM à s'impliquer et à participer aux activités de l'Association des Étudiant·e·s de l'UTBM.

Outre une rétention accrut des jeunes, Instagram donne les chiffres suivants : plus de 140 milliards de Reels visionnés sur Instagram et Facebook chaque jour\footnote{\href{https://business.instagram.com/instagram-reels?locale=fr_FR}{Faites-vous connaître avec Reels}}.

Comme j'utilise le format Reel d'Instagram, je dois me conformer à certaines règles, telles que\footnote{\href{https://about.instagram.com/fr-fr/features/reels}{Reels Instagram}} :
\begin{itemize}
    \item Le format de la vidéo est en 9/16
    \item La durée doit être comprise entre 15 à 90 secondes
    \item Utiliser des musiques libres de droit, ou intégrer directement les musiques proposées par Instagram
    \item Il faut prendre en compte les informations de l'interface Instagram qui peut cacher certains éléments du Reel (voir interface en Annexe la figure : \ref{fig:interfaceInsta})
\end{itemize}


Je peux lister différentes méthodes :
\begin{itemize}
    \item Vidéo d'illustration (avec ou sans bruit du fond)
\end{itemize}

\subsection{Diagramme de Gantt}\label{subsec:diagramme-de-gantt}


\textcolor{red}{Planning initiale  (Gantt) + Commentaire en fin de projet}





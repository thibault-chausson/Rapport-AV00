%! Author = thibaultchausson
%! Date = 23/07/2023

\subsection{Sujet}\label{subsec:sujet}

Le but de ce projet audiovisual est de réaliser une vidéo à la minière d'un article de journal télévisé, pour montrer la diversité de l'engagement associatif, mais plus particulièrement l'engagement des étudiant·e·s.

Le projet montrera divers moments de vie estudiantine, des interviews d'étudiant·e·s, soit en plaine actions, ou non.
Une voix off, que je réaliserai, permettra d'introduire et d'illustrer certaines images.

\subsection{Objectifs}\label{subsec:objectifs}

J'ai pour objectif de donner envie aux autres étudiant·e·s de s'engager dans un projet associatif, soit étudiants ou non, voire de promouvoir l'engagement bénévole au sens large.

De plus, cette courte vidéo pourrait être diffusée plus largement par l'UTBM et l'Association des Étudiant·e·s de l'UTBM pour inciter les nouveaux·elles et ancien·nes de l'UTBM à rejoindre les diverses associations de l'UTBM, voire en créer de nouvelles, je pense particulière à une association humanitaire ou une engagée dans l'intégration des femmes\ldots \
Du fait, que l'associatif manque clairement d'engouement à l'UTBM.

Dans une moindre mesure, ce projet pourrait montrer à l'ensemble du corps enseignant et à l'administration de l'UTBM que l'association est bien plus qu'un regroupement de personnes faisant des soirées, mais plutôt un lieu ou les étudiant·e·s apprennent des compétences transverses au métier de l'ingénieur.
Telle que la gestion de groupe, la gestion de budget, la communication, mais aussi la réalisation de cahier des charges et la gestion de client·e·s, ou de partenaires.

\subsection{Structure}\label{subsec:structure}

Comme dit plus haut, je souhaite réaliser un court documentaire sur l'engagement des étudiant·e·s comme pourrait le faire les \href{https://www.tf1.fr/tf1/grands-reportages}{Grands Reportages}\footnote{Tous les week-ends, Anne-Claire Coudray propose un grand reportage sur ceux qui font l'actualité.} de TF1.
Et enchainer de manière chronologique sur les événements, comme le fait très bien TF1 avec ces émissions

Voici une idée de structure :
\begin{enumerate}
    \item Commencer par parler de l'engagement associatif en avec une voix off, et par exemple des images du live de recrutement P23, voire une introduction plus plateau, et l'accueil des étudiants
    \item Présenter les coulisses de l'intégration, interviews de Florian, image de la soirée d'accueil (avec une courte présentation du responsable : Florian)
    \item Montrer la rentrée à la Junior UTBM avec les formations obligatoires demandées par la \gls{CNJE} (avec une courte présentation du vice-président : Antonin)
    \item Illustrer la première réunion du bureau de l'AE avec une courte présentation du président : Alexis
    \item Passer au BDS avec les premiers créneaux de sport
    \item Faire un gros point sur le déroulement de l'intégration
    \item Côté Junior, montrer les réunion la gestion des intervenants et les réunions client·e·s, voire des activitées réalisées
    \item Pour le BDS, suivre une Nuit du Sport par exemple.
    \item Conclusion, parler de ce qui peut se faire dans d'autres écoles pour ouvrir le chant des possibles, sur l'humanitaire\ldots
    \item Le mot de la fin peut être laissé à Alexis
\end{enumerate}

Ceci n'est qu'une idée de structure, elle pourra évoluer en fonction des images prisse au cours du semestre.

\subsection{Méthodes}\label{subsec:methodes}

Je peux lister différentes méthodes :
\begin{itemize}
    \item Vidéo d'illustration (avec ou sans bruit du fond)
    \item Voix off
    \item Interviews en pleine action
    \item Interview avec du recul, dans un endroit plus calme
    \item Utilisation de vidéo tournée par d'autres personnes (marginal)
\end{itemize}

\subsection{Diagramme de Gantt}\label{subsec:diagramme-de-gantt}







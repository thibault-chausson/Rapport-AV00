%! Author = thibaultchausson
%! Date = 12/09/2023

\subsection{Un retour d'expérience}\label{subsec:un-retour-d'experience}

Ce projet a été une expérience enrichissante à plusieurs niveaux.
J'ai eu l'opportunité de mettre en pratique mes connaissances théoriques dans un contexte réel, confronté aux imprévus et aux exigences du terrain.
La gestion de projet, de la conception à la réalisation, m'a permis d'appréhender concrètement les défis de la communication digitale et de l'engagement associatif.
J'ai également appris l'importance de l'adaptabilité et de la créativité dans la résolution de problèmes, ainsi que la valeur du travail d'équipe et de la collaboration.

D'un point de vue plus académique, l'accompagnement proposé et l'ouverture d'esprit du corps enseignant, qui s'intéresse aux problématiques des étudiant·e·s, est extrêmement appréciable.
Je souhaite profondément réduire la barrière entre enseignant·e·s et étudiant·e·s, en rappelant que chaque personne de l'\gls{UTBM} est conviée aux activités de l'AE, telles que les repas de pôles, ou même sur les lieux de vie pour partager un café.

\subsection{Les compétences développées}\label{subsec:les-competences-developpees}

Ce projet m'a permis de développer plusieurs compétences clés :
\begin{itemize}
    \item \textbf{Compétences techniques :} J'ai acquis une maîtrise approfondie des outils de création et de montage vidéo, ainsi que des techniques de communication sur les réseaux sociaux, en particulier Instagram.
    \item \textbf{Gestion de projet :} J'ai appris à planifier, organiser et exécuter un projet complexe, en gérant efficacement le temps et les ressources.
    \item \textbf{Compétences en communication :} J'ai amélioré mes aptitudes en communication, tant à l'écrit qu'à l'oral, en développant des contenus attractifs et en interagissant avec divers publics.
    \item \textbf{Travail en équipe :} J'ai renforcé ma capacité à travailler en équipe, en collaborant avec différent·e·s acteur·trice·s, en coordonnant les efforts et en respectant les perspectives de chacun.
    \item \textbf{Adaptabilité et résolution de problèmes :} J'ai appris à m'adapter rapidement aux changements et à trouver des solutions créatives aux défis rencontrés.
\end{itemize}


En conclusion, ce projet a été une étape significative dans ma formation, m'offrant des compétences pratiques et une meilleure compréhension des dynamiques de communication et d'engagement associatif, qui seront des compétences que je pourrai valoriser en tant que futur ingénieur.

De plus, bâtir une communauté sur Instagram représente un véritable défi.
Ce réseau social étant très saturé, il exige une stratégie de contenu innovante et captivante pour se démarquer.
En outre, les algorithmes favorisent la nouveauté et l'interaction, ce qui dessert une association telle que l'\gls{AE} qui l'utilise comme canal de communication, bien que de moins en moins mis en avant.
De ce fait, les publications sont mises au second plan, au profit des autres abonnements des étudiant·e·s de l'\gls{UTBM} qui sont jugés par l'algorithme comme plus intéressants et captivants, ce qui entraine une meilleure rétention et ainsi plus de profits pour Instagram.
Il est donc essentiel de créer des publications qui correspondent aux intérêts et aux attentes des étudiant·e·s de l'\gls{UTBM}, tout en restant fidèle à l'identité de l'association.


\begin{center}
    \textit{\og En forçant les hommes à s'occuper d'autre chose que de leurs propres affaires, il combat l'égoïsme individuel, qui est comme la rouille des sociétés. \fg{}}

    Alexis De Tocqueville - \textit{De la Démocratie en Amérique}\cite{De_la_Democratie_en_Amerique}
\end{center}

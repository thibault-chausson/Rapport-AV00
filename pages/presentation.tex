%! Author = thibaultchausson
%! Date = 23/07/2023

Depuis mes 17 ans, je suis membre du bureau directeur du club Rethel Château Canoë Kayak\footnote{\href{https://rcck-ardennes.fr}{RCCK} un club sportif et de loisir de kayak dans le Sud-Ardennes.}.
Il me paraissait inconcevable de ne pas m'engager dans les associations estudiantines de l'UTBM, durant mon cursus d'élève ingénieur.
De ce fait, j'ai rejoint l'Association des Étudiant·e·s de l'UTBM au cours du semestre P23 en tant que Responsable des Relations Extérieures, ainsi que la Junior UTBM au cours de ce même semestre, où j'ai eu la charge de suivre des missions en informatique.
J'ai à regrès décidé d'arrêter mon engagement à l'AE, pour me concentrer sur mon double diplôme avec le Master ETI, malgrè tout, j'ai décidé de rester au sein de la Junior UTBM où j'occuperai le poste de Responsable Qualité durant les semestres A23 et P24.
En ce qui me concerne, l'associatif est une experience humaine très enrichissante, où nous apprenons sans cesse des autres et partageons de forts moments de complicité.

L'association faisant partie intégrante de ma vie et de beaucoup de personne qui m'entoure, c'est en toute logique que j'ai décidé de réaliser mon projet d'audiovisuel sur ce domaine.
En réalisant une courte vidéo montrant la richesse et la diversité de la vie estudiantine, je souhaite la promouvoir et inciter de plus en plus d'étudiant·e·s à la rejoindre les prochaines semestres.

De ce fait, je vais réaliser au cours de ce semestre une vidéo de promotion de cette vie parallèle aux différents impératifs scolaires.
J'ai eu l'occasion suivre de l'intérieur la plus grosse activité de l'Association des Étudiant·e·s, en tant que commis de cuisine pour l'Intégration de l'UTBM.
Outre ces trois semaines intenses, la vie associative et l'engagement étudiant ne s'arrête jamais, le Bureau Des Sports (BDS) organise des créneaux sportifs quotidiennement, la Junior UTBM propose régulièrement des missions rémunérées et professionnalisantes à l'ensemble des étudiant·e·s de l'UTBM.
Bien sûr, les domaines des associations ne se limitent pas qu'aux soirées, aux, sports, ou à l'entrepreneuriat, ce que nous verrons au travers de ce cours documentaire.
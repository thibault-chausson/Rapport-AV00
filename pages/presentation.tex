%! Author = thibaultchausson
%! Date = 23/07/2023

Depuis mes 17 ans, je suis membre du bureau directeur du club Rethel Château Canoë Kayak\footnote{\href{https://rcck-ardennes.fr}{RCCK}, un club sportif et de loisir de kayak dans le Sud-Ardennes.}.
Il m'a paru inconcevable de ne pas m'engager dans les associations estudiantines de l'\gls{UTBM} durant mon cursus d'élève ingénieur.
De ce fait, j'ai rejoint l'Association des Étudiant·e·s de l'\gls{UTBM} au cours du semestre P23 en tant que Responsable des Relations Extérieures, ainsi que la Junior \gls{UTBM} durant ce même semestre, où j'ai eu la charge de suivre des missions en informatique.
À regrets, j'ai décidé d'arrêter mon engagement à l'\gls{AE} pour me concentrer sur mon double diplôme avec le Master ETI. Malgré tout, j'ai décidé de rester au sein de la Junior \gls{UTBM}, où j'occupe le poste de Responsable Qualité durant le semestre A23.
En ce qui me concerne, l'associatif est une expérience humaine très enrichissante, où nous apprenons sans cesse des autres et partageons de forts moments de complicité.

L'association faisant partie intégrante de ma vie et de celle de nombreuses personnes qui m'entourent, c'est en toute logique que j'ai décidé de réaliser mon projet audiovisuel sur ce domaine.
En réalisant ces courtes vidéos montrant la richesse et la diversité de la vie estudiantine, je souhaite la promouvoir et inciter davantage d'étudiant·e·s à la rejoindre dans les prochains semestres.

Ainsi, je vais réaliser, au cours de ce semestre, des vidéos de promotion de cette vie parallèle aux différents impératifs scolaires.
Cette année, j'ai eu l'occasion de suivre de l'intérieur la plus grosse activité de l'Association des Étudiant·e·s, en tant que commis de cuisine pour l'Intégration de l'\gls{UTBM}.
Outre ces trois semaines intenses, la vie associative et l'engagement étudiant ne s'arrêtent jamais : le Bureau Des Sports (BDS) organise des créneaux sportifs quotidiennement, la Junior \gls{UTBM} propose régulièrement des missions rémunérées et professionnalisantes à l'ensemble des étudiant·e·s de l'\gls{UTBM}.
Bien sûr, les domaines des associations ne se limitent pas seulement aux soirées, aux sports, ou à l'entrepreneuriat.